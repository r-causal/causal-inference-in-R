% Options for packages loaded elsewhere
\PassOptionsToPackage{unicode}{hyperref}
\PassOptionsToPackage{hyphens}{url}
%
\documentclass[
]{krantz}
\usepackage{lmodern}
\usepackage{amsmath}
\usepackage{ifxetex,ifluatex}
\ifnum 0\ifxetex 1\fi\ifluatex 1\fi=0 % if pdftex
  \usepackage[T1]{fontenc}
  \usepackage[utf8]{inputenc}
  \usepackage{textcomp} % provide euro and other symbols
  \usepackage{amssymb}
\else % if luatex or xetex
  \usepackage{unicode-math}
  \defaultfontfeatures{Scale=MatchLowercase}
  \defaultfontfeatures[\rmfamily]{Ligatures=TeX,Scale=1}
\fi
% Use upquote if available, for straight quotes in verbatim environments
\IfFileExists{upquote.sty}{\usepackage{upquote}}{}
\IfFileExists{microtype.sty}{% use microtype if available
  \usepackage[]{microtype}
  \UseMicrotypeSet[protrusion]{basicmath} % disable protrusion for tt fonts
}{}
\makeatletter
\@ifundefined{KOMAClassName}{% if non-KOMA class
  \IfFileExists{parskip.sty}{%
    \usepackage{parskip}
  }{% else
    \setlength{\parindent}{0pt}
    \setlength{\parskip}{6pt plus 2pt minus 1pt}}
}{% if KOMA class
  \KOMAoptions{parskip=half}}
\makeatother
\usepackage{xcolor}
\IfFileExists{xurl.sty}{\usepackage{xurl}}{} % add URL line breaks if available
\IfFileExists{bookmark.sty}{\usepackage{bookmark}}{\usepackage{hyperref}}
\hypersetup{
  pdftitle={Causal Inference in R},
  pdfauthor={Lucy D'Agostino McGowan, Malcolm Barrett, Travis Gerke},
  hidelinks,
  pdfcreator={LaTeX via pandoc}}
\urlstyle{same} % disable monospaced font for URLs
\usepackage{graphicx}
\makeatletter
\def\maxwidth{\ifdim\Gin@nat@width>\linewidth\linewidth\else\Gin@nat@width\fi}
\def\maxheight{\ifdim\Gin@nat@height>\textheight\textheight\else\Gin@nat@height\fi}
\makeatother
% Scale images if necessary, so that they will not overflow the page
% margins by default, and it is still possible to overwrite the defaults
% using explicit options in \includegraphics[width, height, ...]{}
\setkeys{Gin}{width=\maxwidth,height=\maxheight,keepaspectratio}
% Set default figure placement to htbp
\makeatletter
\def\fps@figure{htbp}
\makeatother
\setlength{\emergencystretch}{3em} % prevent overfull lines
\providecommand{\tightlist}{%
  \setlength{\itemsep}{0pt}\setlength{\parskip}{0pt}}
\setcounter{secnumdepth}{-\maxdimen} % remove section numbering
\usepackage{hyperref}
\ifluatex
  \usepackage{selnolig}  % disable illegal ligatures
\fi

\title{Causal Inference in R}
\author{Lucy D'Agostino McGowan, Malcolm Barrett, Travis Gerke}
\date{2021-03-11}

\begin{document}
\maketitle

\hypertarget{welcome}{%
\section*{Welcome}\label{welcome}}
\addcontentsline{toc}{section}{Welcome}

Part 1: Asking Causal Questions

\begin{itemize}
\tightlist
\item
  Chapter 1: What is a causal question?

  \begin{itemize}
  \tightlist
  \item
    Description, prediction, and explanation\\
  \item
    Causal assumptions\\
  \item
    Whole game example
  \end{itemize}
\item
  Chapter 2: Expressing causal questions as DAGs

  \begin{itemize}
  \tightlist
  \item
    Visualizing causal assumptions\\
  \item
    DAGs in R: ggdag and dagitty
  \end{itemize}
\item
  Chapter 3: Preparing data to answer causal questions

  \begin{itemize}
  \tightlist
  \item
    Data wrangling with dplyr\\
  \item
    Recognizing missing data: visdat, tidyr, mice\\
  \item
    Working with multiple data sources
  \end{itemize}
\item
  Chapter 4: Observational data as causes and effects

  \begin{itemize}
  \tightlist
  \item
    Exploring and visualizing data and assumptions: skimr, ggplot2\\
  \item
    Calculating summary statistics: gtsummary, tableone
  \end{itemize}
\end{itemize}

Part 2: The counterfactual framework

\begin{itemize}
\tightlist
\item
  Chapter 5: Estimating counterfactuals

  \begin{itemize}
  \tightlist
  \item
    What is a counterfactual?\\
  \item
    Target trials\\
  \item
    Estimating the average treatment effect\\
  \item
    Estimating treatment effects with other targets in mind
  \end{itemize}
\item
  Chapter 6 Building a propensity score models

  \begin{itemize}
  \tightlist
  \item
    Logistic regression\\
  \item
    Choosing variables to include\\
  \item
    Continuous and categorical exposures
  \end{itemize}
\item
  Chapter 7: Using the propensity score

  \begin{itemize}
  \tightlist
  \item
    Matching\\
  \item
    Weighting\\
  \item
    Weighting and matching with more complex exposures
  \end{itemize}
\item
  Chapter 8: Evaluating your propensity score model

  \begin{itemize}
  \tightlist
  \item
    Calculating the standardized mean difference\\
  \item
    Visualizing balance via Love Plots, boxplots, and eCDF plots\\
  \item
    Pruning, trimming, and stabilizing propensity scores
  \end{itemize}
\end{itemize}

Part 3. Estimating causal effects

\begin{itemize}
\tightlist
\item
  Chapter 9: Incorporating propensity scores in generalized linear
  models

  \begin{itemize}
  \tightlist
  \item
    Using matched data sets\\
  \item
    Using weights in outcome models\\
  \item
    Estimating uncertainty\\
  \item
    Estimating causal effects for complex exposures
  \end{itemize}
\item
  Chapter 10: Incorporating propensity scores in survival models

  \begin{itemize}
  \tightlist
  \item
    Preparing data for survival analysis\\
  \item
    Pooled logistic regression\\
  \item
    Confidence intervals for causal survival models
  \end{itemize}
\item
  Chapter 11: Sensitivity analyses

  \begin{itemize}
  \tightlist
  \item
    Quantitative bias analyses\\
  \item
    Tipping point analyses: tipr, EValue
  \end{itemize}
\item
  Chapter 12: Other approaches to causal inference

  \begin{itemize}
  \tightlist
  \item
    G-computation\\
  \item
    Targeted Learning\\
  \item
    Instrumental variable analysis\\
  \item
    Regression discontinuity\\
  \item
    Difference-in-Difference
  \end{itemize}
\end{itemize}

\end{document}
